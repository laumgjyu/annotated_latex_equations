\begin{wrapfigure}{l}{0.5\columnwidth}
    \vspace{\baselineskip}
    \begin{equation}
    \label{eq:laplace_density}
        \lap (x\ |\ \tikzmarknode{u}{\highlight{red}{$\mu$}}, \tikzmarknode{b}{\highlight{blue}{b}}) = \frac{1}{2b} \mathrm{exp}(-\frac{|x-\mu|}{b}) 
    \end{equation}
    \begin{tikzpicture}[overlay,remember picture,>=stealth,nodes={align=left,inner ysep=1pt},<-]
        % For "mu"
        \path (u.north) ++ (0,2em) node[anchor=south west,color=red!67] (scalep){\textbf{location parameter, mean}};
        \draw [color=red!57](u.north) |- ([xshift=-0.3ex,color=red]scalep.south east);
        % For "b"
        \path (b.south) ++ (0,-1.5em) node[anchor=north west,color=blue!67] (mean){\textbf{$b >0$, scale parameter}};
        \draw [color=blue!57](b.south) |- ([xshift=-0.3ex,color=blue]mean.south east);
    \end{tikzpicture}
    \vspace{0.5\baselineskip}
    \caption{An example in the single column format using the wrapfig construct.}
    \vspace{0.5\baselineskip}
\end{wrapfigure}

% 下面解释一下上述代码

% 首先要明白node的概念. node可以看做是一个点, 可以与其他点连线.
% 定义一个node: node[params] (name) {description}. 
% params表示这个node的参数, 如shape, anchor, color. shape表示node的形状, 可以将node设置为圆形 (circle), 矩形 (rectangle, 默认) 或坐标 (coordinate); anchor表示将这个node哪个部分放置到这个点所在的坐标上, 默认是center, 即将node的中间部分放到所设置的坐标上, 可选参数很多, 详见node_anchor.jpg; color表示这个node的description的颜色.
% name为定义的这个node的名称, 类似函数变量名. 在后面可以通过这个名称引用这个node.
% description表示这个node上要显示的文字描述.

% 上面第9行就定义了一个node: node[anchor=south west,color=red!67] (scalep){\textbf{location parameter, mean}}
% 其中, anchor参数表明了将该node的左下角放置在所要设置坐标上, color表明node的描述为红色, scalep为这个node的名称, 后面的中括号里是这个node的描述.

% 第5行使用了\tikzmarknode{name}{content}来基于现有的content定义一个node. 
% name为这个node的名称, content为这个node的内容, 这个node的位置就是content的位置. 
% 例如, 第5行中, \tikzmarknode{u}{\highlight{red}{$\mu$}}, u为这个node的名称, \highlight{red}{$\mu$}为这个node的内容或description, 这个node的位置就是其内容所在的位置.


% 其次要了解\path的概念.
