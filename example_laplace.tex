\begin{wrapfigure}{l}{0.5\columnwidth}
    \vspace{\baselineskip}
    \begin{equation}
    \label{eq:laplace_density}
        \lap (x\ |\ \tikzmarknode{u}{\highlight{red}{$\mu$}}, \tikzmarknode{b}{\highlight{blue}{b}}) = \frac{1}{2b} \mathrm{exp}(-\frac{|x-\mu|}{b}) 
    \end{equation}
    \begin{tikzpicture}[overlay,remember picture,>=stealth,nodes={align=left,inner ysep=1pt},<-]
        % For "mu"
        \path (u.north) ++ (0,2em) node[anchor=south west,color=red!67] (scalep){\textbf{location parameter, mean}};
        \draw [color=red!57](u.north) |- ([xshift=-0.3ex,color=red]scalep.south east);
        % For "b"
        \path (b.south) ++ (0,-1.5em) node[anchor=north west,color=blue!67] (mean){\textbf{$b >0$, scale parameter}};
        \draw [color=blue!57](b.south) |- ([xshift=-0.3ex,color=blue]mean.south east);
    \end{tikzpicture}
    \vspace{0.5\baselineskip}
    \caption{An example in the single column format using the wrapfig construct.}
    \vspace{0.5\baselineskip}
\end{wrapfigure}

% 下面解释一下上述代码

% 首先要明白node的概念. node可以看做是一个点, 可以与其他点连线.
% 定义一个node: node[params] (name) {description}. 
% params表示这个node的参数, 如shape, anchor, color. shape表示node的形状, 可以将node设置为圆形 (circle), 矩形 (rectangle, 默认) 或坐标 (coordinate); anchor表示将这个node哪个部分放置到这个点所在的坐标上, 默认是center, 即将node的中间部分放到所设置的坐标上, 可选参数很多, 详见node_anchor.jpg; color表示这个node的description的颜色.
% name为定义的这个node的名称, 类似函数变量名. 在后面可以通过这个名称引用这个node.
% description表示这个node上要显示的文字描述.

% 上面第9行就定义了一个node: node[anchor=south west,color=red!67] (scalep){\textbf{location parameter, mean}}
% 其中, anchor参数表明了将该node的左下角放置在所要设置坐标上, color表明node的描述为红色, scalep为这个node的名称, 后面的中括号里是这个node的描述.

% 第5行使用了\tikzmarknode{name}{content}来基于现有的content定义一个node. 
% name为这个node的名称, content为这个node的内容, 这个node的位置就是content的位置. 
% 例如, 第5行中, \tikzmarknode{u}{\highlight{red}{$\mu$}}, u为这个node的名称, \highlight{red}{$\mu$}为这个node的内容或description, 这个node的位置就是其内容所在的位置.


% 其次要了解\draw的概念.
% \draw表示划线. 其后通常跟着坐标, 如: \drarw (0,0) -- (1,1); 表示从坐标(0, 0)画一条到坐标(1, 1)的直线. 不写单位默认为cm.
% 可以写多个坐标, 通过\draw在这多个坐标之间连线. 
% 坐标可写成偏移的形式: \draw (0,0) -- +(1,1) -- +(0,1); % (0,0) -- (0+1,0+1) -- (0+0,0+1);
% 使用一个+表示偏移时, 每个坐标的偏移都是基于第一个坐标的; 使用两个+, 即++, 表示偏移时, 每个坐标的偏移都是基于前一个坐标的, 例如: \draw (0,0) -- ++(1,1) -- +(0,1); %(0,0) -- (0+1,0+1) -- (0+1+0, 0+1+1);
% 坐标还可以替换为node的名称.
% 两个坐标之间的连线形式: --表示两点之间连直线; |-表示两点之间先画竖线, 再画直线; -|表示两点之间先画直线, 再画竖线.


% 现在可以了解\path的概念了. 这里只介绍本项目中用到的\path的用法.
% \path在不加任何参数时, 是画一条显示不出来的线. 因此, 本项目利用\path来给node定位.
% 例如第9行: \path (u.north) ++ (0,2em) node[anchor=south west,color=red!67] (scalep){\textbf{location parameter, mean}};
% 表示, 从u这个node的上面 (north), 纵坐标距离u 2em, 横坐标与u相同的位置, 放置一个名为scalep的node. 如果以u为坐标原点, 那么scalep的左下角位于 (0, 2em) 的位置.


% 通过介绍上述三个概念, 就可以看懂该项目的代码, 并基于该项目改写.
% 本文件中, 第5行在公式的几个项上放置node; 
% 第9行, 在u这个node的north这个锚点的上面2em处, 创建一个新的node, 名为scalep, 内容为\textbf{location parameter, mean}, 内容的颜色为red!67, node的左下角放置在u.north上面2em处的点上;
% 第10行: \draw [color=red!57](u.north) |- ([xshift=-0.3ex,color=red]scalep.south east);
% 先看个简化版的第10行: \draw (u.north) |- (scalep.south east); 将node u的north锚点和node scalep的south east锚点相连, 连线为折线, 从u.north开始, 先画竖线, 再画横线, 最后与scalep的south east锚点相连接.
% 在第10行, \draw的参数表示折线的颜色为red!57, scalep前的中括号表示对这个node的参数进行修改, 其中xshift表示将这个node沿横轴移动.
